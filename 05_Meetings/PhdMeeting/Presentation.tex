%%%%%%%%%%%%%%%%%%%%%%%%%%%%%%%%%%%%%%%%%
% Beamer Presentation
% LaTeX Template
% Version 1.0 (01/07/19)
%
%%%%%%%%%%%%%%%%%%%%%%%%%%%%%%%%%%%%%%%%%

%----------------------------------------------------------------
%	PACKAGES AND THEMES		-----------------------------------
%----------------------------------------------------------------

\documentclass[xcolor=table,11pt]{beamer}

\mode<presentation> {
	
	\usetheme{Frankfurt}
	\usecolortheme{dove}
	\usefonttheme{serif}
	
}
\usepackage{newtxtext,newtxmath}
\usepackage{graphicx}
\usepackage{booktabs} 
\usepackage{subfig}
\usepackage{pgf}
\usepackage{multirow}
\usepackage{appendixnumberbeamer}
\usepackage{bookmark}
\usepackage{siunitx}
\usepackage{animate}
\usepackage{xcolor}
\usepackage{soul}
\usepackage{pifont}
\usepackage{caption}
\usepackage{array}
\captionsetup{skip=0pt,belowskip=0pt}


%----------------------------------------------------------------
%	GENERAL OPTIONS 	-----------------------------------------
%----------------------------------------------------------------

% Set template options
\setbeamertemplate{section in toc}{\inserttocsectionnumber.~\inserttocsection}
\setbeamertemplate{frametitle}{\vspace*{1em}\insertframetitle}
\setbeamertemplate{enumerate items}[default]
\setbeamercolor{section in head/foot}{fg=white, bg=black}

% Headline
\makeatletter
\setbeamertemplate{headline}
{%
	\pgfuseshading{beamer@barshade}%
	\vskip-5ex%
	\begin{beamercolorbox}[ignorebg,ht=2.25ex,dp=3.75ex]{section in head/foot}
		\insertsectionnavigationhorizontal{\paperwidth}{\hskip0pt plus1fill}{\hskip0pt plus1fill}
	\end{beamercolorbox}%
	\ifbeamer@sb@subsection%
	\begin{beamercolorbox}[ignorebg,ht=2.125ex,dp=1.125ex,%
		leftskip=.3cm,rightskip=.3cm plus1fil]{subsection in head/foot}
		\usebeamerfont{subsection in head/foot}\insertsubsectionhead
	\end{beamercolorbox}%
	\fi%
}%
\makeatother

% Footline
\makeatletter
\setbeamertemplate{footline}
{
	\leavevmode%
	\hbox{%
		\begin{beamercolorbox}[wd=.333333\paperwidth,ht=2.25ex,dp=1ex,left]{section in head/foot}%
			\usebeamerfont{author in head/foot}\hspace{10pt}\insertshortauthor
		\end{beamercolorbox}%
		\begin{beamercolorbox}[wd=.333333\paperwidth,ht=2.25ex,dp=1ex,center]{section in head/foot}%
			\usebeamerfont{title in head/foot}\insertshorttitle
		\end{beamercolorbox}%
		\begin{beamercolorbox}[wd=.333333\paperwidth,ht=2.25ex,dp=1ex,right]{section in head/foot}%
			\usebeamerfont{date in head/foot}\insertshortdate{}\hspace*{2em}
			\insertframenumber{}\hspace*{2em}
	\end{beamercolorbox}}%
	\vskip0pt%
}
\makeatother

% Add logo
\logo{\pgfputat{\pgfxy(0,7)}{\pgfbox[right,base]{\includegraphics[width=0.1\paperwidth]{Pictures/00_Unibe_Logo}}}}

% Table settings
\renewcommand{\arraystretch}{2}
\captionsetup{labelformat=empty,labelsep=none}
\definecolor{Gray}{gray}{0.9}

% Define highlitghing command
\makeatletter
\let\HL\hl
\renewcommand\hl{%
	\let\set@color\beamerorig@set@color
	\let\reset@color\beamerorig@reset@color
	\HL}
\makeatother

% Add overview at each begin of section
%\AtBeginSection[]
%{
	%	\begin{frame}
		%		\frametitle{Overview}
		%		\tableofcontents[currentsection]
		%	\end{frame}
	%}


\renewcommand{\arraystretch}{1.4}
\newcommand{\ColWidth}{1}
\newcommand{\TrimSize}{50}

%----------------------------------------------------------------
%	TITLE PAGE 	-------------------------------------------------
%----------------------------------------------------------------

\title[PhD Meeting]{Limitations of Homogenized Finite Elements Analysis of Distal Tibia Sections} 

\author{Mathieu Simon}
\institute[University of Bern]
{
Phd Student - Musculoskeletal Biomechanics \\
Supervisor: Philippe Zysset \\
}
\medskip
\date{April 18, 2023}

\begin{document}
	
	\begin{frame}
		\titlepage
	\end{frame}
	
	%----------------------------------------------------------------
	%----------------------------------------------------------------
	%----------------------------------------------------------------
	
	\section{Introduction}

	\begin{frame}
		\frametitle{The Disease - Osteoporosis}

		\begin{columns}
			\column{0.48\linewidth}
			Loss of bone mass
			\begin{itemize}
				\item Increase of fracture risk \cite{p1}
			\end{itemize}

			\vspace{5mm}

			\includegraphics[height=0.7\linewidth]{Pictures/BoneMass}\\
			\begin{center}
				\tiny{Adapted from \cite{p3}}
			\end{center}

			\column{0.48\linewidth}
			8.9 millions fractures per year \cite{p2}
			\begin{itemize}
				\item A fracture every 3 seconds
			\end{itemize}

			\vspace{5mm}

			\includegraphics[height=0.7\linewidth]{Pictures/White}\\
			\begin{center}
				\tiny{\textcolor{white}{Image from eurospinepatientline.org}}
			\end{center}
		\end{columns}

	\end{frame}

	\begin{frame}[noframenumbering]
		\frametitle{The Disease - Osteoporosis}

		\begin{columns}
			\column{0.48\linewidth}
			Loss of bone mass
			\begin{itemize}
				\item Increase of fracture risk \cite{p1}
			\end{itemize}

			\vspace{5mm}

			\includegraphics[height=0.7\linewidth]{Pictures/BoneMass1}\\
			\begin{center}
				\tiny{Adapted from \cite{p3}}
			\end{center}

			\column{0.48\linewidth}
			8.9 millions fractures per year \cite{p2}
			\begin{itemize}
				\item A fracture every 3 seconds
			\end{itemize}

			\vspace{5mm}

			\includegraphics[height=0.7\linewidth]{Pictures/Osteoporosis1}\\
			\begin{center}
				\tiny{Image from eurospinepatientline.org}
			\end{center}
		\end{columns}

	\end{frame}

	\begin{frame}[noframenumbering]
		\frametitle{The Disease - Osteoporosis}

		\begin{columns}
			\column{0.48\linewidth}
			Loss of bone mass
			\begin{itemize}
				\item Increase of fracture risk \cite{p1}
			\end{itemize}

			\vspace{5mm}

			\includegraphics[height=0.7\linewidth]{Pictures/BoneMass2}\\
			\begin{center}
				\tiny{Adapted from \cite{p3}}
			\end{center}

			\column{0.48\linewidth}
			8.9 millions fractures per year \cite{p2}
			\begin{itemize}
				\item A fracture every 3 seconds
			\end{itemize}

			\vspace{5mm}

			\includegraphics[height=0.7\linewidth]{Pictures/Osteoporosis2}\\
			\begin{center}
				\tiny{Image from eurospinepatientline.org}
			\end{center}

		\end{columns}

	\end{frame}

	\begin{frame}
		\frametitle{Osteoporosis - Diagnosis}

		Gold standard - DXA
		\begin{itemize}
			\item Areal bone mineral density (aBMD)
		\end{itemize}

		\begin{columns}
			\column{0.4\linewidth}
			\centering
			\includegraphics[width=\linewidth]{Pictures/DXA}\\
			\tiny{\cite{p4}}

			\column{0.55\linewidth}
			\centering
			\includegraphics[width=\linewidth]{Pictures/BMDTscore}\\
			\tiny{\cite{p5}}

		\end{columns}

	\end{frame}



	\begin{frame}
		\frametitle{HR-pQCT-based Diagnostic}

		\begin{columns}
			\column{0.65\linewidth}
			Bone 3D structure
			\begin{itemize}
				\item Assess microarchitecture
				\item Differentiate deterioration \cite{p6}
			\end{itemize}
	
			\vspace{10mm}
	
			Micro-finite element (\si{\mu}FE)
			\begin{itemize}
				\item Direct voxel converstion to hexahedral elements
			\end{itemize}

			\column{0.30\linewidth}
			\centering
			\includegraphics[width=\linewidth]{Pictures/HRpQCT}

		\end{columns}

	\end{frame}

	\begin{frame}
		\frametitle{Homogenized finite element (hFE)}

		\begin{columns}
			\column{0.60\linewidth}
			
				"Average" properties:
				\begin{itemize}
					\item Fabric
					\item Bone volume fraction (BV/TV)
				\end{itemize}

				\vspace{5mm}

				Good prediction \cite{p13}
				\begin{itemize}
					\item Stiffness
					\item Ultimate Load
				\end{itemize}
	
			\vspace{5mm}
			Is hFE also good for strain localization ?\\$\Rightarrow$ Varga et al. \cite{p7}

			\column{0.35\linewidth}
			\includegraphics[width=\linewidth]{Pictures/Fabric}\\
		\end{columns}
	\end{frame}


	%----------------------------------------------------------------
	%----------------------------------------------------------------
	%----------------------------------------------------------------
	
	\section{Methods}

	\begin{frame}
		\frametitle{Summary}
		25 samples\\
		\vfill
		\centering
		\includegraphics[width=1.0\linewidth]{Pictures/Methods}
	\end{frame}

	\begin{frame}
		\frametitle{Imaging - Pre-test \si{\mu}CT}
		\begin{columns}
			\column{0.45\linewidth}

			\column{0.45\linewidth}
			\centering
			\vspace{3.9mm}\\
			\includegraphics[width=1.0\linewidth]{Pictures/PreTest}
		\end{columns}
	\end{frame}

	\begin{frame}[noframenumbering]
		\frametitle{Imaging - Pre-test \si{\mu}CT}
		\begin{columns}
			\column{0.45\linewidth}
			Segmented Image
			\begin{columns}
				\column{0.85\linewidth}
				\begin{itemize}
					\item Bone morphometry
				\end{itemize}
				\column{0.1\linewidth}
			\end{columns}

			\vspace{51.5mm}

			\column{0.45\linewidth}
			\centering
			\includegraphics[width=1.0\linewidth]{Pictures/Seg}
		\end{columns}
	\end{frame}

	\begin{frame}[noframenumbering]
		\frametitle{Imaging - Pre-test \si{\mu}CT}
		\begin{columns}
			\column{0.45\linewidth}
			Segmented Image
			\begin{columns}
				\column{0.85\linewidth}
				\begin{itemize}
					\item Bone morphometry
				\end{itemize}
				\column{0.1\linewidth}
			\end{columns}

			\vspace{5mm}

			Filled mask
			\begin{columns}
				\column{0.45\linewidth}
				\begin{itemize}
					\item Volume
					\item Height
				\end{itemize}
				\column{0.5\linewidth}
				\begin{itemize}
					\item Mean area
				\end{itemize}
			\end{columns}

			\vspace{30.75mm}

			\column{0.45\linewidth}
			\centering
			\includegraphics[width=1.0\linewidth]{Pictures/Mask}
		\end{columns}
	\end{frame}

	\begin{frame}[noframenumbering]
		\frametitle{Imaging - Pre-test \si{\mu}CT}
		\begin{columns}
			\column{0.45\linewidth}
			Segmented Image
			\begin{columns}
				\column{0.85\linewidth}
				\begin{itemize}
					\item Bone morphometry
				\end{itemize}
				\column{0.1\linewidth}
			\end{columns}

			\vspace{5mm}

			Filled mask
			\begin{columns}
				\column{0.45\linewidth}
				\begin{itemize}
					\item Volume
					\item Height
				\end{itemize}
				\column{0.5\linewidth}
				\begin{itemize}
					\item Mean area
				\end{itemize}
			\end{columns}

			\vspace{5mm}

			Gray values
			\begin{columns}
				\column{0.95\linewidth}
				\begin{itemize}
					\item Volumetric bone mineral density (vBMD)
					\item Bone mineral content (BMC)
				\end{itemize}
				\column{0.05\linewidth}
			\end{columns}

			\column{0.45\linewidth}
			\centering
			\includegraphics[width=1.0\linewidth]{Pictures/PreTest}
		\end{columns}
	\end{frame}

	\begin{frame}
		\frametitle{Compressive Test}
		\begin{columns}
			\column{0.35\linewidth}
			Extensive properties
			\begin{itemize}
				\item Stiffness
				\item Ultimate Force
			\end{itemize}

			\vspace{5mm}

			Intensive properties
			\begin{itemize}
				\item Apparent modulus
				\item Apparent strength
			\end{itemize}


			\column{0.6\linewidth}
			\centering
			\hspace{-10mm}
			\includegraphics[width=0.35\linewidth]{Pictures/Test}
			\includegraphics[width=0.75\linewidth]{Pictures/ForceDisp}
		\end{columns}
	\end{frame}

	%----------------------------------------------------------------
	%----------------------------------------------------------------
	%----------------------------------------------------------------
	
	\section{Results}

	\begin{frame}
		\frametitle{Structural Response}
		\centering
		\vfill
		\begin{columns}
			\column{0.45\linewidth}
			\centering
			Extensive properties\\
			\vspace{2mm}
			\includegraphics[width=0.8\linewidth, trim=5 0 0 0]{Pictures/SvsBMC}\\
			\includegraphics[width=0.8\linewidth, trim=0 0 0 25]{Pictures/UvsBMC}

			\column{0.45\linewidth}
			\centering
			Intensive propoerties\\
			\vspace{2mm}
			\includegraphics[width=0.8\linewidth, trim=0 0 0 0]{Pictures/MvsBMD}\\
			\includegraphics[width=0.8\linewidth, trim=5 0 0 25]{Pictures/SvsBMD}
		\end{columns}
		\vfill
	\end{frame}

	\begin{frame}
		\frametitle{hFE vs Mechanical Test}
		\centering
		\vfill
		\begin{columns}
			\column{0.45\linewidth}
			\centering
			Stiffness\\
			\includegraphics[width=1.0\linewidth]{Pictures/Stiffness}

			\column{0.45\linewidth}
			\centering
			Ultimate load\\
			\includegraphics[width=1.0\linewidth]{Pictures/UltimateLoad}
		\end{columns}
	\end{frame}

	\begin{frame}
		\frametitle{hFE vs Registration}
		\centering
		\vfill
		\begin{columns}
			\column{0.33\linewidth}
			\vfill
			3 categories\\
			\vfill
			\begin{itemize}
				\item Good agreement\\12\% ( 3/25)
				\item \textcolor{white}{Partial agreement\\40\% (10/25)}
				\item \textcolor{white}{Bad agreement\\48\% (12/25)}
			\end{itemize}
			\vfill
			\vspace{2.5mm}

			\column{0.4\linewidth}
			\centering
			hFE\\
			\includegraphics[height=0.5\linewidth]{Pictures/hFE3}\vspace{5mm}\\
			Registration\\
			\includegraphics[height=0.5\linewidth]{Pictures/Reg3}\\
			\vspace{2.5mm}

		\end{columns}
	\end{frame}

	\begin{frame}[noframenumbering]
		\frametitle{hFE vs Registration}
		\centering
		\vspace{2mm}
		\begin{columns}
			\column{0.33\linewidth}
			\vfill
			3 categories\\
			\vfill
			\begin{itemize}
				\item \textcolor{gray}{Good agreement\\12\% ( 3/25)}
				\item Partial agreement\\40\% (10/25)
				\item \textcolor{white}{Bad agreement\\48\% (12/25)}
			\end{itemize}
			\vfill

			\column{0.4\linewidth}
			\centering
			hFE\\
			\includegraphics[height=0.5\linewidth]{Pictures/hFE2}\vspace{5mm}\\
			Registration\\
			\includegraphics[height=0.5\linewidth]{Pictures/Reg2}
		\end{columns}
	\end{frame}

	\begin{frame}[noframenumbering]
		\frametitle{hFE vs Registration}
		\centering
		\vspace{2mm}
		\begin{columns}
			\column{0.33\linewidth}
			\vfill
			3 categories\\
			\vfill
			\begin{itemize}
				\item \textcolor{gray}{Good agreement\\12\% ( 3/25)}
				\item \textcolor{gray}{Partial agreement\\40\% (10/25)}
				\item Bad agreement\\48\% (12/25)
			\end{itemize}
			\vfill

			\column{0.4\linewidth}
			\centering
			hFE\\
			\includegraphics[height=0.5\linewidth]{Pictures/hFE1}\vspace{5mm}\\
			Registration\\
			\includegraphics[height=0.5\linewidth]{Pictures/Reg1}
		\end{columns}
	\end{frame}

	%----------------------------------------------------------------
	%----------------------------------------------------------------
	%----------------------------------------------------------------
	
	\section{Discussion and Conclusion}

	\begin{frame}
		\frametitle{Discussion}

		Structural response
		\begin{itemize}
			\item Good to very good R$^2$ (0.81 to 0.93)
			\item Similar to radius \cite{p8}\cite{p9}
			\item Similar to vertebral bodies \cite{p10}\cite{p11}
		\end{itemize}

		\vfill

		hFE predictions
		\begin{itemize}
			\item Excellent for stiffness (R$^2$ = 0.95) and ultimate load (R$^2$ = 0.97)\\Similar to radius \cite{p9}, radius and tibia \cite{p13}\\Similar to vertebral bodies \cite{p10}
			\item Strain localization\\Contrast with Varga et al. \cite{p7}
		\end{itemize}

	\end{frame}
	
	\begin{frame}
		\frametitle{Conclusion - hFE}
		Excellent for stiffness and ultimate load prediction for
		\begin{itemize}
			\item Vertebral bodies
			\item Distal radius
			\item Distal tibia
		\end{itemize}
		
		\vspace{5mm}

		Limitations in strain localization prediction
		\begin{itemize}
			\item hFE operate at millimeter scale\\$\rightarrow$ Mechanical failure occurs at micro-meter scale
			\item Importance of intact distal segment\\$\rightarrow$ Intact cortical shell
			\item Importance of segment length\\$\rightarrow$ St-Venant principle
		\end{itemize}

	\end{frame}
	
	%----------------------------------------------------------------
	%----------------------------------------------------------------
	%----------------------------------------------------------------
	
	\section{}
	\begin{frame}
		\frametitle{Aknowledgements}
		Funds:
		\begin{itemize}
			\item SNSF
			\item ARTORG Center for Biomedical Engineering Research
		\end{itemize}
		\vfill
		Samples
		\begin{itemize}
			\item Division of Anatomy\\Center for Anatomy and Cell Biology\\Medical University of Vienna\\Vienna, Austria
		\end{itemize}
		\vfill
		Compression testing
		\begin{itemize}
			\item AO Research Institute Davos\\7270 Davos, Switzerland
		\end{itemize}
	\end{frame}
	
	%----------------------------------------------------------------
	%----------------------------------------------------------------
	%----------------------------------------------------------------
	
	\appendix
	
	\section{References}

	\begin{frame}
		\frametitle{Introduction}
		\footnotesize{
				\begin{thebibliography}{99}
						\setbeamertemplate{bibliography item}[triangle]
						
						\bibitem[1]{p1} Sozen, T., Ozisik, L., Basaran, N. C. (2017)
						\newblock An overview and management of osteoporosis
						\newblock \textit{Eur J Rheumatol}, 4(1), 46-56

						\bibitem[2]{p2}Johnell, O., Kanis, J. A. (2006)
						\newblock An estimate of the worldwide prevalence and disability associated with osteoporotic fractures
						\newblock \textit{Osteoporos Int}, 17(12), 1726-1733

						\bibitem[3]{p3} Cooper C., Melton L.J. (1992)
						\newblock Epidemiology of osteoporosis
						\newblock \textit{Trends Endocrinol Metab}, 3(6), 224-229
						
					\end{thebibliography}
			}
	\end{frame}

	\begin{frame}
		\frametitle{Introduction}
		\footnotesize{
				\begin{thebibliography}{99}
						\setbeamertemplate{bibliography item}[triangle]
						
						\bibitem[4]{p4} Oei, L., Koromani, F., Rivadeneira, F., Zillikens, M. C. (2016)
						\newblock Quantitative imaging methods in osteoporosis
						\newblock \textit{Quantitative Imaging in Medicine and Surgery}, 6(6), 680-698
						
						\bibitem[5]{p5}  Ferrari, S. L., Roux, C. (2019)
						\newblock Pocket Reference to Osteoporosis
						\newblock \textit{Springer International Publishing}

						\bibitem[6]{p6}  van den Bergh JP, Szulc P, Cheung AM, Bouxsein M, Engelke K, Chapurlat R. (2021)
						\newblock The clinical application of high-resolution peripheral computed tomography (HR-pQCT) in adults: state of the art and future directions
						\newblock \textit{Osteoporos Int}, 32(8), 1465-1485

						\bibitem[7]{p7}  Varga, P., Baumbach, S., Pahr, D., Zysset, P. K. (2009)
						\newblock Validation of an anatomy specific finite element model of Colles' fracture.
						\newblock \textit{Journal of Biomechanics}, 42(11), 1726-1731
						
					\end{thebibliography}
			}
	\end{frame}

	\begin{frame}
		\frametitle{Discussion}
		\footnotesize{
				\begin{thebibliography}{99}
						\setbeamertemplate{bibliography item}[triangle]

						\bibitem[8]{p8} Varga, P., Pahr, D. H., Baumbach, S., Zysset, P. K. (2010)
						\newblock HR-pQCT based FE analysis of the most distal radius section provides an improved prediction of Colles' fracture load in vitro
						\newblock \textit{Bone}, 47(5), 982-988

						\bibitem[9]{p9} Varga, P., Dall'Ara, E., Pahr, D. H., Pretterklieber, M., Zysset, P. K. (2011)
						\newblock Validation of an HR-pQCT-based homogenized finite element approach using mechanical testing of ultra-distal radius sections
						\newblock \textit{Biomechanics and Modeling in Mechanobiology}, 10(4), 431-444

						\bibitem[10]{p10} Dall'Ara, E., Schmidt, R., Pahr, D., Varga, P., Chevalier, Y., Patsch, J., Kainberger, F., Zysset, P. (2010)
						\newblock A nonlinear finite element model validation study based on a novel experimental technique for inducing anterior wedge-shape fractures in human vertebral bodies in vitro.
						\newblock \textit{Journal of Biomechanics}, 43(12), 2374-2380
						
					\end{thebibliography}
			}
	\end{frame}

	\begin{frame}
		\frametitle{Discussion}
		\footnotesize{
				\begin{thebibliography}{99}
						\setbeamertemplate{bibliography item}[triangle]

						\bibitem[11]{p11} Dall'Ara, E., Pahr, D., Varga, P., Kainberger, F., Zysset, P. (2012)
						\newblock QCT-based finite element models predict human vertebral strength in vitro significantly better than simulated DEXA
						\newblock \textit{Osteoporosis International}, 23(2), 563-572

						\bibitem[12]{p12} Arias-Moreno, A. J., Hosseini, H. S., Bevers, M., Ito, K., Zysset, P., Rietbergen, B. van. (2019)
						\newblock Validation of distal radius failure load predictions by homogenized- and micro-finite element analyses based on second-generation high-resolution peripheral quantitative CT images
						\newblock \textit{Osteoporosis International}, 30(7), 1433-1443

						\bibitem[13]{p13}Schenk, D., Indermaur, M., Simon, M., Voumard, B., Varga, P., Pretterklieber, M., Lippuner, K., Zysset, P. (2022)
						\newblock Unified validation of a refined second-generation HR-pQCT based homogenized finite element method to predict strength of the distal segments in radius and tibia
						\newblock \textit{Journal of the Mechanical Behavior of Biomedical Materials}, 131

						
					\end{thebibliography}
			}
	\end{frame}
	
	%----------------------------------------------------------------
	%----------------------------------------------------------------
	%----------------------------------------------------------------
	
\end{document}