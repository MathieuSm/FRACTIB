%%%%%%%%%%%%%%%%%%%%%%%%%%%%%%%%%%%%%%%%%
% Beamer Presentation
% LaTeX Template
% Version 1.0 (01/07/19)
%
%%%%%%%%%%%%%%%%%%%%%%%%%%%%%%%%%%%%%%%%%

%----------------------------------------------------------------
%	PACKAGES AND THEMES		-----------------------------------
%----------------------------------------------------------------

\documentclass[xcolor=table]{beamer}

\mode<presentation> {

\usetheme{Frankfurt}
\usecolortheme{dove}
\usefonttheme{serif}

}
\usepackage{newtxtext,newtxmath}
\usepackage{graphicx}
\usepackage{booktabs} 
\usepackage{subfig}
\usepackage{pgf}
\usepackage{multirow}
\usepackage{appendixnumberbeamer}
\usepackage{bookmark}
\usepackage{siunitx}
\usepackage{animate}
\usepackage{xcolor}
\usepackage{soul}
\usepackage{pifont}
\usepackage{caption}
\captionsetup{skip=0pt,belowskip=0pt}


%----------------------------------------------------------------
%	GENERAL OPTIONS 	-----------------------------------------
%----------------------------------------------------------------

% Set template options
\setbeamertemplate{section in toc}{\inserttocsectionnumber.~\inserttocsection}
\setbeamertemplate{frametitle}{\vspace*{1em}\insertframetitle}
\setbeamertemplate{enumerate items}[default]
\setbeamercolor{section in head/foot}{fg=white, bg=black}

% Headline
\makeatletter
\setbeamertemplate{headline}
{%
  \pgfuseshading{beamer@barshade}%
    \vskip-5ex%
  \begin{beamercolorbox}[ignorebg,ht=2.25ex,dp=3.75ex]{section in head/foot}
  \insertsectionnavigationhorizontal{\paperwidth}{\hskip0pt plus1fill}{\hskip0pt plus1fill}
  \end{beamercolorbox}%
  \ifbeamer@sb@subsection%
    \begin{beamercolorbox}[ignorebg,ht=2.125ex,dp=1.125ex,%
      leftskip=.3cm,rightskip=.3cm plus1fil]{subsection in head/foot}
      \usebeamerfont{subsection in head/foot}\insertsubsectionhead
    \end{beamercolorbox}%
  \fi%
}%
\makeatother

% Footline
\makeatletter
\setbeamertemplate{footline}
{
  \leavevmode%
  \hbox{%
  \begin{beamercolorbox}[wd=.333333\paperwidth,ht=2.25ex,dp=1ex,left]{section in head/foot}%
    \usebeamerfont{author in head/foot}\hspace{10pt}\insertshortauthor
  \end{beamercolorbox}%
  \begin{beamercolorbox}[wd=.333333\paperwidth,ht=2.25ex,dp=1ex,center]{section in head/foot}%
    \usebeamerfont{title in head/foot}\insertshorttitle
  \end{beamercolorbox}%
  \begin{beamercolorbox}[wd=.333333\paperwidth,ht=2.25ex,dp=1ex,right]{section in head/foot}%
    \usebeamerfont{date in head/foot}\insertshortdate{}\hspace*{2em}
    \insertframenumber{}\hspace*{2em}
  \end{beamercolorbox}}%
  \vskip0pt%
}
\makeatother

% Add logo
\logo{\pgfputat{\pgfxy(0,7)}{\includegraphics[width=0.1\paperwidth]{Pictures/00_Unibe_Logo}}}

% Table settings
\renewcommand{\arraystretch}{2}
\captionsetup{labelformat=empty,labelsep=none}
\definecolor{Gray}{gray}{0.9}

% Define highlitghing command
\makeatletter
\let\HL\hl
\renewcommand\hl{%
	\let\set@color\beamerorig@set@color
	\let\reset@color\beamerorig@reset@color
	\HL}
\makeatother

% Add overview at each begin of section
%\AtBeginSection[]
%{
%	\begin{frame}
%		\frametitle{Overview}
%		\tableofcontents[currentsection]
%	\end{frame}
%}

%----------------------------------------------------------------
%	TITLE PAGE 	-------------------------------------------------
%----------------------------------------------------------------

\title[Regular Meeting]{
\uppercase{Regular Meeting}
} 

\author{Mathieu Simon}
\institute[University of Bern]
{
MSc - Biomedical Engineering \\
University of Bern, Faculty of Medicine \\
\medskip
}
\date{May 3, 2021}

\begin{document}

\begin{frame}
\titlepage
\end{frame}

%----------------------------------------------------------------
%----------------------------------------------------------------
%----------------------------------------------------------------

\begin{frame}
	\frametitle{Overview}
	\tableofcontents
\end{frame}

%----------------------------------------------------------------
%----------------------------------------------------------------
%----------------------------------------------------------------

\section{2D Affine Registration}

\begin{frame}
	\frametitle{Test Initialization}
	\begin{columns}
		\column[c]{0.45\linewidth}
		\centering
		\includegraphics[width=1\linewidth,trim=150 -50 150 0]{Pictures/01_FixedImage}\\
		Fixed and moving images
		\column[c]{0.45\linewidth}
		\centering
		\begin{figure}
			\includegraphics[width=1\linewidth,trim=150 -50 150 0]{Pictures/01_MovingImage}
			\caption{Moving Image}
		\end{figure}
	\end{columns}
\end{frame}



\begin{frame}
	\frametitle{Test Results I}
	\begin{columns}
		\column[c]{0.45\linewidth}
		\centering
		\begin{figure}
			\includegraphics[width=1\linewidth,trim=150 -50 150 0]{Pictures/01_FixedImage}
			\caption{Fixed Image}
		\end{figure}
		\column[c]{0.45\linewidth}
		\centering
		\begin{figure}
			\includegraphics[width=1\linewidth,trim=150 -50 150 0]{Pictures/02_DisplacementField}
			\caption{Displacement Field}
		\end{figure}
	\end{columns}
\end{frame}



\begin{frame}
	\frametitle{Test Results II}
	\begin{columns}
		\column[c]{0.45\linewidth}
		\centering
		\begin{figure}
			\includegraphics[width=1\linewidth,trim=150 -50 150 0]{Pictures/03_J}
			\caption{Spherical Compression}
		\end{figure}
		\column[c]{0.45\linewidth}
		\centering
		\begin{figure}
			\includegraphics[width=1\linewidth,trim=150 -50 150 0]{Pictures/03_NormF}
			\caption{Isovolumic Deformation}
		\end{figure}
	\end{columns}
\end{frame}

%----------------------------------------------------------------
%----------------------------------------------------------------
%----------------------------------------------------------------

\section{2D Affine + B-Spline Registration}

\begin{frame}
	\frametitle{Test I Initialization}
	\begin{columns}
		\column[c]{0.45\linewidth}
		\centering
		\begin{figure}
			\includegraphics[width=1\linewidth,trim=150 -50 150 0]{Pictures/04_FixedImage}
			\caption{Fixed Image}
		\end{figure}
		\column[c]{0.45\linewidth}
		\centering
		\begin{figure}
			\includegraphics[width=1\linewidth,trim=150 -50 150 0]{Pictures/04_MovingImage}
			\caption{Moving Image}
		\end{figure}
	\end{columns}
\end{frame}



\begin{frame}
	\frametitle{Test I Results I}
	\begin{columns}
		\column[c]{0.45\linewidth}
		\centering
		\begin{figure}
			\includegraphics[width=1\linewidth,trim=150 -50 150 0]{Pictures/04_FixedImage}
			\caption{Fixed Image}
		\end{figure}
		\column[c]{0.45\linewidth}
		\centering
		\begin{figure}
			\includegraphics[width=1\linewidth,trim=150 -50 150 0]{Pictures/05_DisplacementField}
			\caption{Displacement Field}
		\end{figure}
	\end{columns}
\end{frame}



\begin{frame}
	\frametitle{Test I Results II}
	\begin{columns}
		\column[c]{0.45\linewidth}
		\centering
		\begin{figure}
			\includegraphics[width=1\linewidth,trim=150 -50 150 0]{Pictures/06_J}
			\caption{Spherical Compression}
		\end{figure}
		\column[c]{0.45\linewidth}
		\centering
		\begin{figure}
			\includegraphics[width=1\linewidth,trim=150 -50 150 0]{Pictures/06_NormF}
			\caption{Isovolumic Deformation}
		\end{figure}
	\end{columns}
\end{frame}



\begin{frame}
	\frametitle{Observations}
	\begin{itemize}
		\item Important deformations lead sometimes to negative $J$
		\item $||\tilde{F}||$ highlights high deformation zones
	\end{itemize}
\end{frame}



\begin{frame}
	\frametitle{Test II Initialization}
	\begin{columns}
		\column[c]{0.45\linewidth}
		\centering
		\begin{figure}
			\includegraphics[width=1\linewidth,trim=150 -50 150 0]{Pictures/07_FixedImage}
			\caption{Fixed Image}
		\end{figure}
		\column[c]{0.45\linewidth}
		\centering
		\begin{figure}
			\includegraphics[width=1\linewidth,trim=150 -50 150 0]{Pictures/07_MovingImage}
			\caption{Moving Image}
		\end{figure}
	\end{columns}
\end{frame}



\begin{frame}
	\frametitle{Test II Results I}
	\begin{columns}
		\column[c]{0.45\linewidth}
		\centering
		\begin{figure}
			\animategraphics[controls,width=1\linewidth,trim=-10 0 -10 0]{2}{Pictures/08_Image-}{0}{1}\\
			\caption{Fixed and moving image}
		\end{figure}
		\column[c]{0.45\linewidth}
		\centering
		\begin{figure}
			\includegraphics[width=1\linewidth,trim=150 -50 150 0]{Pictures/08_DisplacementField}
			\caption{Displacement Field}
		\end{figure}
	\end{columns}
\end{frame}



\begin{frame}
	\frametitle{Test II Results II}
	\begin{columns}
		\column[c]{0.45\linewidth}
		\centering
		\begin{figure}
			\includegraphics[width=1\linewidth,trim=150 -50 150 0]{Pictures/09_J}
			\caption{Spherical Compression}
		\end{figure}
		\column[c]{0.45\linewidth}
		\centering
		\begin{figure}
			\includegraphics[width=1\linewidth,trim=150 -50 150 0]{Pictures/09_NormF}
			\caption{Isovolumic Deformation}
		\end{figure}
	\end{columns}
\end{frame}


\begin{frame}
	\frametitle{Test II Results III}
	\begin{columns}
		\column[c]{0.45\linewidth}
		\centering
		\begin{figure}
			\includegraphics[width=1\linewidth,trim=-50 -50 -50 0]{Pictures/10_Damaged}
			\caption{Manually highlighting}
		\end{figure}
		\column[c]{0.45\linewidth}
		\centering
		\begin{figure}
			\includegraphics[width=1\linewidth,trim=150 -50 150 0]{Pictures/10_DeformationMagnitude2}
			\caption{Deformation magnitude}
		\end{figure}
	\end{columns}
\end{frame}



\begin{frame}
	\frametitle{Observations}
	\begin{itemize}
		\item Smaller deformations
		\item Still some negative $J$
		\item $||\tilde{F}||$ and deformation magnitude highlight high deformation zones well
	\end{itemize}
\end{frame}


%----------------------------------------------------------------
%----------------------------------------------------------------
%----------------------------------------------------------------
\appendix

%\section{References}
%\begin{frame}
%	\frametitle{Statistics}
%	\footnotesize{
%		\begin{thebibliography}{99}
%			\setbeamertemplate{bibliography item}[triangle]
%			\bibitem[Fox 2016]{p8} Fox, John (2016).
%			\newblock Fox, Applied Regressions Analysis and Linear Models
%			\newblock \textit{Sage publications}(3), 817
%		\end{thebibliography}
%	}
%\end{frame}

%----------------------------------------------------------------
%----------------------------------------------------------------
%----------------------------------------------------------------

\end{document} 